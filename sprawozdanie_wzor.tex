\documentclass[12pt]{article}
\usepackage[T1]{fontenc}
\usepackage[T1]{polski}
\usepackage[utf8]{inputenc}
\newcommand{\BibTeX}{{\sc Bib}\TeX} 
\usepackage{graphicx}
\usepackage{amsfonts}
\usepackage{listings}
\usepackage{color}

\lstset{frame=tb,
language=R,
keywordstyle=\color{blue},
alsoletter={.}
}

\setlength{\textheight}{21cm}

\title{{\bf Zadanie nr 1 - Klasyfikacja}\linebreak
Programowanie w językach 4GL}
\author{Izabela Pabich, 220088 \and Wojciech Pełka, Nr albumu}
\date{data oddania zadania}

\begin{document}
\clearpage\maketitle
\thispagestyle{empty}
\newpage
\setcounter{page}{1}
\section{Cel zadania}

Celem zadania jest zaimplementowanie w języku R rozwiązania problemu klasyfikacji na wybranym zbiorze danych z dostępnej bazy UCI (odniesienie do biblio). Algorytmem, którego należy użyć do zrealizowania celu jest algorytm k-NN (k najbliższych sąsiadów). \\

\section{Wstęp teoretyczny}

\subsection{Zbiór danych}
Wybranym przez nas zbiorem jest zbiór do klasyfikacji irysów. W zbiorze dostępne są charakterystyki 3 rodzai kwiatów, gdzie każdy z nich opisany jest czterema wartościami. Plik zawiera 150 rekordów, bo 50 każdego z gatunków irysa. By móc przeprowadzić klasyfikację zbiór należy podzielić na zbiór treningowy oraz testowy. \\

\subsection{Algorytm k-NN}
Algorytm k-NN, czyli k najbliższych sąsiadów może zostać użyty do problemu klasyfikacji. Rozwiązanie polega na odnalezieniu \textit{k}, gdzie \textit{k} jest liczbą naturalną najczęściej nieparzystą, najmniej oddalonych obiektów dla punktu przeznaczonego do umieszczenia w klasie ~\cite{dowolna_etykieta_ksiazki}). \\
Metoda początkowo rozmieszcza w przestrzeni wartości zbioru treningowego tworząc bazę według, której będzie umiała przyporządkować kolejne punkty do wydzielonych podzbiorów. Następnie mierząc odległości między punktem testowym a wyznaczoną bazą jest w stanie odnaleźć najbliższe obiekty oraz zakwalifikować wybrane wartości do jednej z klas. W przypadku braku jednoznacznego rozstrzygnięcia przyporządkowanie jest losowe. \\
Odległość między punktami może być liczona na wiele sposobów. W naszym rozwiązaniu jest to odległość Euklidesowa (odniesienie do biblio). \\
Na odpowiednie rozmieszczenie obiektów ze zbiorów ma wpływ wiele czynników takich jak: liczba próbek w zbiorze treningowym, liczba \textit{k}, liczba klas do klasyfikacji. Przeprowadzając pewną liczbę eksperymentów będziemy starać się ustalić parametry, które pozwolą na jak najbardziej dokładną klasyfikację.

\newpage

\section{Eksperymenty i wyniki}

Eksperymenty zostały przeprowadzone przez funkcję \textit{knn()} języka R. Przykładowe jej wykorzystynie przedstawia poniższy listing:\\
{\footnotesize
\begin{lstlisting}
iris_pred <- knn(train=iris_data.training, test=iris_data.test, 
		 cl=iris_data.trainLabels, k=3)
\end{lstlisting}
}

Przyjmuję ona kilka parametrów:
\begin{itemize}
\item \textit{train} - przyjmuje zbiór danych treningowych
\item \textit{test} - przyjmuje zbiór danych testowych
\item \textit{cl} - zawiera zbiór etykiet przypisywanych do wyników klasyfikacji
\item \textit{k} - określa liczbę najbliższych sąsiadów branych pod uwagę
\end{itemize}

Do przetestowania metody k-NN klasyfikacji zbioru zostało przygotowanych kilka przypadków testowych, których wyniki przestawione są poniżej. Zmienialiśmy liczność zbiorów treningowych i testowych oraz parametr \textit{k}.

\subsection{Eksperyment nr 1}
\subsubsection{Założenia}

Parametry eksperymentu nr 1:
\begin{itemize}
\item \textit{k} = \textbf{3}
\item iris\_data.training = \textbf{108}
\item iris\_data.test = \textbf{42}
\end{itemize}

\newpage
\subsubsection{Rezultat}

Rezultaty badań eksperymentalnych przedstawione są w Tab. \ref{wyniki1}.
\begin{table}[ht!]
 \caption{Rezultaty eksperymentu nr 1}
 \centering
 \vspace{0.2cm}
 \begin{tabular}{|c|c|c|c|c|}
  \hline\\[-0.5cm]
  \textbf{Rodzaj kwiatu} & \textbf{Setosa} & \textbf{Versicolor} & \textbf{Virginica} & \textbf{Razem}\\[0.1cm]
  \hline
  \textbf{Setosa} & 13 & 0 & 0 & 13  \\ \hline
  \textbf{Versicolor} & 0 & 13 & 0 & 13  \\ \hline
  \textbf{Virginica} & 0 &  1 & 15 & 16  \\ \hline
  \textbf{Razem} & 13 & 14 & 15 & 42 \\ [0.1cm]
  \hline
 \end{tabular}
 \label{wyniki1}
\end{table}

\noindent Jak widać w Tab. \ref{wyniki1} wyniki w 97,6\% dają poprawne rezultaty. Maszyna pomyliła się tylko w 1 na 42 przypadki. \newline
%Graficzna interpretacja wynik�w z Tab. \ref{wyniki eksperymentu pierwszego} 
%przedstawiona jest na wykresie Rys. \ref{rysunek do eksperymentu pierwszego} gdzie mo�na zauwa�y�, �e...
%\begin{figure}[h!]
% \centering
% \includegraphics[width=9.3cm]{wykres.pdf}
% \vspace{-0.3cm}
% \caption{Wykres dla wynik�w eksperymentu pierwszego}
% \label{rysunek do eksperymentu pierwszego}
%\end{figure}
%
%\noindent Jak wida� z wykresu Rys. \ref{rysunek do eksperymentu pierwszego}...\newline
%
%%%%%%%%%%%%%%%%%%%%%%%%%%%%%%%%%%%%%%%%%%%%%%%%%%%%%%%%%%%%%%%%%%%%%%%%%%%%%%%%%%%%%%%%%%%%%%%%%%%%%%%%%%%%%%%%%%
%% PODROZDZIA� PT. EKSPERYMENT NR 2 
%%%%%%%%%%%%%%%%%%%%%%%%%%%%%%%%%%%%%%%%%%%%%%%%%%%%%%%%%%%%%%%%%%%%%%%%%%%%%%%%%%%%%%%%%%%%%%%%%%%%%%%%%%%%%%%%%%
%
%\newpage
%\subsection{Eksperyment nr 2}
%
%Eksperyment nr 2 polega� na...\\
%Sigmoidalna funkcja aktywacji ma posta�:
%\begin{equation}
% \forall\: s \in \mathbb{R}\:\:\:\:\:\: f(s) = \frac{1}{\:\:\:1 + e^{-\beta \cdot s}\:}\:,
% \:\:\:\:\textnormal{gdzie}\:\:\beta \in \mathbb{R}_{+}
% \label{r�wnanie funkcji sigmoidalnej}
%\end{equation}
%Jak wida� z r�wnania definicyjnego (\ref{r�wnanie funkcji sigmoidalnej}) 
%funkcja\footnote{ang. \textit{sigmoidal function} lub \textit{unipolar function}}
%ta ma wykres przedstawiony 
%na rysunku Rys. \ref{funkcja sigmoidalna}, gdzie paramater $\beta$ ...
%\begin{figure}[h!]
% \centering
% \includegraphics[width=7.3cm]{funkcja.png}
% \vspace{-0.1cm}
% \caption{Wykres funkcji sigmoidalnej}
% \label{funkcja sigmoidalna}
%\end{figure}
%
%\subsubsection{Za�o�enia}
%
%\subsubsection{Przebieg}
%
%\subsubsection{Rezultat}
%
%Rezultaty bada� eksperymentalnych przedstawione s� w Tab. \ref{wyniki eksperymentu drugiego}.
%\vspace{-0.5cm}
%\begin{table}[h!]
% \caption{Rezultaty eksperymentu nr 2}
% \centering
% \vspace{0.2cm}
% \begin{tabular}{c c c }
%  \hline\hline\\[-0.4cm]
%  \textbf{Przypadek} & \textbf{Metoda 1} & \textbf{Metoda 2}\\[0.1cm]
%  \hline
%  \textbf{1} & 50 & 837 \\
%  \textbf{2} & 47 & 877 \\
%  \textbf{3} & 45 & 300 \\ [0.1cm]
%  \hline
% \end{tabular}
% \label{wyniki eksperymentu drugiego}
%\end{table}
%
%\noindent Jak wida� w Tab. \ref{wyniki eksperymentu drugiego}...\newline
%Wyniki w Tab. \ref{wyniki eksperymentu drugiego} �wiadcz� o tym, �e...
%
%%%%%%%%%%%%%%%%%%%%%%%%%%%%%%%%%%%%%%%%%%%%%%%%%%%%%%%%%%%%%%%%%%%%%%%%%%%%%%%%%%%%%%%%%%%%%%%%%%%%%%%%%%%%%%%%%%
%% PODROZDZIA� PT. EKSPERYMENT NR N 
%%%%%%%%%%%%%%%%%%%%%%%%%%%%%%%%%%%%%%%%%%%%%%%%%%%%%%%%%%%%%%%%%%%%%%%%%%%%%%%%%%%%%%%%%%%%%%%%%%%%%%%%%%%%%%%%%%
%
%\subsection{Eksperyment nr n}
%Eksperyment nr n zak�ada�, i�...\\
%Dla dowolnej liczby $N \in \mathbb{N}$ funkcj� 
%$F_{N}:\mathbb{C}^{\:N}\!\rightarrow\mathbb{C}^{\:N}$
%zdefiniowan� w nast�puj�cy spos�b:
%\vspace{-0.4cm}
%\begin{equation}
% \forall\:\mathbf{x} \in \mathbb{C}^{\:N}\:\:\:\:\:
% \forall\:k \in \{\:0,\dots,N - 1\:\!\}\:\:\:\:\:
% F_{N}\:\!(\:\mathbf{x}\:)_{k}\:\stackrel{\Delta}{=}\:
% \frac{1}{\sqrt{N}}\:
% \sum_{n = 0}^{N - 1}\:x_{n}\:\cdot\:
% e^{-j 2 \pi n k / N}
% \label{r�wnanie dyskretnej transformaty Fouriera}
%\end{equation}
%nazywamy $N$ -- punktowym prostym jednowymiarowym dyskretnym przekszta�ceniem Fouriera.
%Na Rys. \ref{FFT} przedstawiono szybki algorytm obliczania dyskretnego 
%przekszta�cenia Fouriera\footnote{ang. \textit{Fast Fourier Transform}}.
%\begin{figure}[h!]
% \centering
% \includegraphics[width=0.8\linewidth]{transformata.pdf}
% \vspace{-0.1cm}
% \caption{Szybkie przekszta�cenie Fouriera}
% \label{FFT}
%\end{figure}
%
%\subsubsection{Za�o�enia}
%
%\subsubsection{Przebieg}
%
%\subsubsection{Rezultat}
%\newpage

\section{Wnioski}

%Wnioski z przeprowadzonych eksperyment�w dowodz�, �e...
%
%

\renewcommand\refname{Bibliografia}
\bibliographystyle{plain}
\bibliography{bibliografia}

\end{document}
